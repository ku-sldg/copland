\documentclass[12pt]{report}
\usepackage[T1]{fontenc}
\usepackage{lmodern}
%\usepackage{fullpage}
\usepackage{coqdoc}
\usepackage{amsmath,amssymb}
\usepackage{url}
\usepackage{hyperref}

\iffalse
\title{Attestation Protocol Description Terms:\\
  Semantics and Coq Proofs\thanks{Approved for Public Release;
    Distribution Unlimited. Public Release Case Number 18-3576.}}

\author{John D. Ramsdell\qquad\qquad Paul D. Rowe\\[3ex]
  The MITRE Corporation}
\fi

\begin{document}

\begin{titlepage}
  \vspace*{7ex}
  \begin{center}\LARGE
    Attestation Protocol Description Terms:\\
    Semantics and Coq Proofs
  \end{center}
  \vspace{3ex}
  \begin{center}\Large
    John D. Ramsdell\qquad\qquad Paul D. Rowe\\[3ex]
    The MITRE Corporation
  \end{center}
  \vfill
  The view, opinions, and/or findings contained in this report are
  those of The MITRE Corporation and should not be construed as an
  official Government position, policy, or decision, unless designated
  by other documentation.
  \noindent\copyright 2018 The MITRE Corporation. This technical data
  deliverable was developed using contract funds under Basic Contract
  No. W15P7T-13-C-A802.
  \noindent Approved for Public Release;
  Distribution Unlimited. Public Release Case Number 18-3576.
\end{titlepage}

%\maketitle

\begin{abstract}
  Attestation Protocol Description Terms (APDTs) provide a means for
  specifying layered attestations.  The terms are designed to bridge
  the gap between formal analysis of attestation security guarantees
  and concrete implementations.  We therefore provide two semantic
  interpretations of terms in our language.  The first is a
  denotational semantics in terms of partially ordered sets of
  events.  This directly connects APDTs to prior work on layered
  attestation.  The second is an operational semantics detailing how
  the data and control flow are executed.  This gives explicit
  implementation guidance for attestation frameworks.

  This document is generated from Coq sources that contain the proofs
  of the connection between the two semantics ensuring that any
  execution according to the operational semantics is consistent with
  the denotational event semantics.  This ensures that formal
  guarantees resulting from analyzing the event semantics will hold
  for executions respecting the operational semantics.
\end{abstract}

\tableofcontents

\coqlibrary{Introduction}{}{Introduction}

Attestation Protocol Description Terms (APDTs) provide a means for
specifying layered attestations.  The terms are designed to bridge the
gap between formal analysis of attestation security guarantees and
concrete implementations.  We therefore provide two semantic
interpretations of terms in our language.  The first is a denotational
semantics in terms of partially ordered sets of events.  This directly
connects APDTs to prior work on layered attestation.  The second is an
operational semantics detailing how the data and control flow are
executed.  This gives explicit implementation guidance for attestation
frameworks.

This document is generated from a set of proof scripts for the Coq
proof assistant.  Chapter 2 contains a few tactics that are used
throughout the proofs that follow.  Chapter 3 contains facts about
generic lists in Coq.  Many of the lemmas are in support of particular
proofs, so the motivation for some is obscure.  A notable exception
is the definition and lemmas about whether an element $x$ is
\emph{earlier} than $y$ in list $\ell$.  This will be used to discuss
event orderings in traces.  A trace is a list of events.

Chapter 4 precisely specifies APDTs and the events that are generated
when an APDT is executed.  The script also defines annotated terms.
To properly distinguish events, each event is associated with a
natural number.  Annotated terms are used to produce unique natural
numbers for events.

Chapter 5 shows our representation of strict partially ordered sets of
abstract events.  The events are only required to have a function that
produces its natural number.  Proofs include a demonstration that the
relation used to order events is, in fact, a strict partial order.
Chapter 6 specializes the event system to the case of APDTs and their
events.

Chapter 7 defines a big-step semantics for APDTs.  The semantics
associates a term, a place, and some initial evidence with a trace.
The semantics is specified inductively, mirroring the structure of
APDTs.  The chapter concludes by showing that the order of events in a
trace specified by the big-step semantics is compatible with the
partial order given by the APDT event system.

Chapter 8 defines a small-step semantics for APDTs using a labeled
transition system (LTS).  The proofs in this chapter demonstrate that
the LTS (1) computes the correct evidence associated with a term; (2) can
always proceed unless it is in a halt state; and (3) always
terminates.

Chapter 9 contains a proof of the main theorem of this work: a trace
generated by the small-step semantics is compatible with the partial
order given by the associated APDT event system.  The main lemma is
that every trace generated by the small-step semantics is a trace of
the big-step semantics.

\input{body}

\end{document}
